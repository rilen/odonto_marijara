```
\documentclass[a4paper,12pt]{article}
\usepackage[utf8]{inputenc}
\usepackage[T1]{fontenc}
\usepackage{lmodern}
\usepackage{geometry}
\geometry{margin=1in}
\usepackage{graphicx}
\usepackage{hyperref}
\usepackage{enumitem}
\usepackage{xcolor}
\usepackage{titlesec}
\titleformat{\section}{\large\bfseries}{\thesection}{1em}{}
\titleformat{\subsection}{\normalsize\bfseries}{\thesubsection}{1em}{}

\begin{document}

\title{\textbf{Sistema Web para Consultório Odontológico - Documentação}}
\author{Clínica Odontológica - Marijara}
\date{20 de Maio de 2025}
\maketitle

\section{Visão Geral}
Este documento descreve o sistema web desenvolvido para consultórios odontológicos, com módulos para gestão de pacientes, agendamentos, financeiro, estoque, relatórios, odontograma, notificações, dashboard, configurações, anamnese, treinamento, suporte e avaliação de pacientes. O sistema é construído com React, Vite, e Tailwind CSS, com suporte offline via PWA.

\section{Requisitos Técnicos}
\begin{itemize}
    \item \textbf{Navegador}: Chrome, Firefox, ou Edge (versões recentes).
    \item \textbf{Dependências}: Node.js 18+, npm.
    \item \textbf{Hospedagem}: Vercel, AWS Elastic Beanstalk, MongoDB Atlas, Cloudinary.
\end{itemize}

\section{Módulos do Sistema}
\subsection{Acesso}
\begin{itemize}
    \item Gerenciamento de contas (admin, dentista, operador).
    \item Bloqueio após 5 tentativas de login.
\end{itemize}

\subsection{Contatos}
\begin{itemize}
    \item Cadastro de pacientes, fornecedores, dentistas.
    \item Busca rápida por CPF ou nome.
\end{itemize}

\subsection{Agendamento}
\begin{itemize}
    \item Calendário interativo (FullCalendar).
    \item Lembretes via e-mail/WhatsApp.
\end{itemize}

\subsection{Financeiro}
\begin{itemize}
    \item Contas a pagar/receber.
    \item Relatórios de faturamento.
\end{itemize}

\subsection{Estoque}
\begin{itemize}
    \item Controle com QR codes.
    \item Alertas de estoque baixo.
\end{itemize}

\subsection{Relatorios}
\begin{itemize}
    \item Relatórios de consultas, financeiro, estoque.
    \item Exportação em PDF/CSV.
\end{itemize}

\subsection{Odontograma}
\begin{itemize}
    \item Interface SVG para 32 dentes.
    \item Exportação em PDF.
\end{itemize}

\subsection{Notificacoes}
\begin{itemize}
    \item Lembretes por e-mail, WhatsApp, push.
\end{itemize}

\subsection{Dashboard}
\begin{itemize}
    \item KPIs: consultas, faturamento, satisfação.
\end{itemize}

\subsection{Configuracoes}
\begin{itemize}
    \item Personalização de tema, fuso horário, idioma.
\end{itemize}

\subsection{Anamnese}
\begin{itemize}
    \item Formulário com assinatura digital.
    \item Upload de exames.
\end{itemize}

\subsection{Treinamento}
\begin{itemize}
    \item Tutoriais com gamificação.
    \item Certificados em PDF.
\end{itemize}

\subsection{Suporte}
\begin{itemize}
    \item Chat e tickets.
    \item FAQs com IA simulada.
\end{itemize}

\subsection{Avaliacao}
\begin{itemize}
    \item Avaliações pós-consulta.
    \item Gráficos de satisfação.
\end{itemize}

\section{Instruções de Instalação}
\begin{enumerate}
    \item Clonar repositório: \texttt{git clone <url>}.
    \item Instalar dependências: \texttt{npm install}.
    \item Iniciar servidor local: \texttt{npm run dev}.
    \item Deploy no Vercel: Vincular repositório e configurar como Vite.
\end{enumerate}

\section{Observações}
\begin{itemize}
    \item Para produção, configurar MongoDB Atlas, AWS S3, Cloudinary.
    \item Testar PWA offline após primeiro carregamento.
\end{itemize}

\end{document}
```
